\documentclass{article}
\usepackage[utf8]{inputenc}

\usepackage[letterpaper, portrait, margin=1in]{geometry}
\usepackage{array}
\usepackage{amssymb}
\usepackage{tabularx}
\usepackage{amsthm}
\usepackage{amsmath}
\renewcommand{\qedsymbol}{\rule{0.7em}{0.7em}}

\title{Math 115A Discussion 1}
\author{Mateo Umaguing, Matthew Sasaki, Zifeng Zhou, Ziteng Liu, Zhuoya Li}
\date{September 2021}

\begin{document}

\maketitle

\subsubsection*{ I. \textbf{Background for proofs.}}

\subsubsection*{ II. \textbf{Examples of fields.}}
\begin{enumerate}
    \item Prove that $\mathbb{Q}[\sqrt{2}]$ is a field.
    \begin{enumerate}
        \item FS4. Additive inverse and multiplicative inverse \\
        Proof 1: Additive Inverse
        \begin{displaymath}
            \forall a, b, a + \sqrt{2} b \in \mathbb{Q}[\sqrt{2}], \ 
            a + \sqrt{2}b - a - \sqrt{2}b = 0
        \end{displaymath}
        \begin{displaymath}
            \textrm{ Since } -a \in \mathbb{Q} \textrm{ and } -b \in \mathbb{Q},
            -a - \sqrt{2}b \in \mathbb{Q}[\sqrt{2}].
        \end{displaymath}
        \begin{displaymath}
            \textrm{Since } -a - \sqrt{2}b \in \mathbb{Q}[\sqrt{2}],
            a + \sqrt{2} b \in \mathbb{Q}[\sqrt{2}] \textrm{ has an additive inverse.}
        \end{displaymath}
        Proof 2: Multiplicative Inverse
        \begin{displaymath}
            \forall a, b, a + \sqrt{2} b \in \mathbb{Q}[\sqrt{2}], \ 
            a + \sqrt{2}b * \frac{1}{a - \sqrt{2}b} = 1
        \end{displaymath}
        \begin{displaymath}
            \frac{1}{a - \sqrt{2}b} * \frac{a - \sqrt{2}b}{a - \sqrt{2}b} = \frac{a - \sqrt{2}b}{a^2 - 2b^2} = \frac{a}{a^2 - 2b^2} + \frac{(-1)\sqrt{2}b}{a^2 + 2b^2}
        \end{displaymath}
        \begin{displaymath}
            \frac{a}{a^2 + 2b^2} \in \mathbb{Q} \textrm{ and } \frac{(-1)b}{a^2 - 2b^2} \in \mathbb{Q}, \textrm{ therefore }
        \end{displaymath}
        \begin{displaymath}
            \frac{a}{a^2 - 2b^2} + \frac{(-1)\sqrt{2}b}{a^2 + 2b^2} \in \mathbb{Q}[\sqrt{2}]
        \end{displaymath}
        \begin{displaymath}
            \textrm{Since } \frac{1}{a - \sqrt{2}b} \in \mathbb{Q}[\sqrt{2}], \ 
            a + \sqrt{2}b \textrm{ has a multiplicative inverse } \forall a,b
        \end{displaymath}
        
    \end{enumerate}
\end{enumerate}

\subsubsection*{ III. \textbf{Computational problems.}}
\begin{enumerate}
    \item Transposed matrices
    \begin{displaymath}
        A = \begin{pmatrix}
            1 & 0 & 3 \\
            0 & 2 & 0 \\
            3 & 0 & 1
        \end{pmatrix}, \ 
        A^{T} = \begin{pmatrix}
            1 & 0 & 3 \\
            0 & 2 & 0 \\
            3 & 0 & 1
        \end{pmatrix}
    \end{displaymath}
    \begin{displaymath}
        B = \begin{pmatrix}
            74 & 10,802 & 13 \\
            0 & 1 & 81
        \end{pmatrix}, \ 
        B^{T} = \begin{pmatrix}
            74 & 0 \\
            10,802 & 1 \\
            13 & 81
        \end{pmatrix}
    \end{displaymath}
    \begin{displaymath}
        C = \begin{pmatrix}
            1 & 2 & 3 & 4 \\
            2 & 0 & 0 & 0 \\
            3 & 0 & 0 & 0 \\
        \end{pmatrix}, \ 
        C^{T} = \begin{pmatrix}
            1 & 2 & 3 \\
            2 & 0 & 0 \\
            3 & 0 & 0 \\
            4 & 0 & 0 \\
        \end{pmatrix}
    \end{displaymath}
    \item Matrix sum $A + B$ and matrix product $AB$
    \begin{enumerate}
        \item Polynomial entries
        \begin{displaymath}
            A + B = \begin{pmatrix}
                x + 3x^3 + 1 & 2x^2 + 2 \\
                6 - 5x & 9 + x
            \end{pmatrix}
        \end{displaymath}
        \begin{displaymath}
            AB = \begin{pmatrix}
                3x^3 + 10x^2 + x & 2x^3 + 14x^2 + 2 \\
                -15x^4 + 3x^3 - 5x^2 + x + 10 & -8x + 16
            \end{pmatrix}
        \end{displaymath}
        \item Entries in $\mathbb{Z} / 2\mathbb{Z}$
        \begin{displaymath}
            A + B = \begin{pmatrix}
                1 & 0 \\
                1 & 0
            \end{pmatrix}
        \end{displaymath}
        \begin{displaymath}
            AB = \begin{pmatrix}
                1 & 0 \\
                1 & 1
            \end{pmatrix}
        \end{displaymath}
    \end{enumerate}
\end{enumerate}

\end{document}
